\documentclass[11pt]{article}
\usepackage[margin = 1in]{geometry}
\usepackage{amsmath}
\usepackage{amssymb}
\usepackage{amsthm}
\usepackage{graphicx}
\usepackage{enumitem}
\usepackage{url}
\usepackage[parfill]{parskip}
\usepackage{listings}
\usepackage{caption}
\usepackage{subcaption}
\usepackage{mdframed}
\usepackage[utf8]{inputenc}
\usepackage{xcolor}
\definecolor{codegreen}{rgb}{0,0.6,0}
\definecolor{codegray}{rgb}{0.5,0.5,0.5}
\definecolor{codepurple}{rgb}{0.58,0,0.82}
\definecolor{backcolour}{rgb}{0.95,0.95,0.92}
\lstdefinestyle{mystyle}{
	backgroundcolor=\color{backcolour},   
	commentstyle=\color{codegreen},
	keywordstyle=\color{magenta},
	numberstyle=\tiny\color{codegray},
	stringstyle=\color{codepurple},
	basicstyle=\ttfamily\footnotesize,
	breakatwhitespace=false,         
	breaklines=true,                 
	captionpos=b,                    
	keepspaces=true,                 
	numbers=left,                    
	numbersep=5pt,                  
	showspaces=false,                
	showstringspaces=false,
	showtabs=false,                  
	tabsize=2
}
\lstset{style=mystyle}
\usepackage{hyperref}
\hypersetup{
	colorlinks=true,
	linkcolor=blue,
	filecolor=magenta,      
	urlcolor=red,
	pdfpagemode=FullScreen,
}
\newcommand{\skipline}{\vspace{\baselineskip}}
\newcommand{\spacer}{\noalign{\medskip}}
\newcommand{~}{\sim}
\newcommand{\approches}{\rightarrow}
\newcommand{\qarrow}{\quad \rightarrow \quad}
\newcommand{\qqarrow}{\qquad \rightarrow \qquad}
\newcommand{\qtext}[1]{\quad \text{ #1 } \quad}
\newcommand{\qqtext}[1]{\qquad \text{ #1 } \qquad}
\newcommand{\pard}[2]{\frac{\partial #1}{\partial #2}}
\newcommand{\answer}[1]{\textbf{\boldmath #1}}
\newenvironment{problem}[1]{\textbf{Excersise #1: }}{\newpage}
\newcommand{\mol}{\text{ mol }}
\begin{document}
	
	\begin{center}
		\textbf{Chemistry Review} \\
		\textbf{Stephen Giang} \\
		\skipline \skipline
	\end{center}

	\textbf{Links: }
	\begin{enumerate}[label=(\alph*)]
		\item \href{https://www.youtube.com/watch?v=5yw1YH7YA7c&ab_channel=TheOrganicChemistryTutor}{General Chemistry 1 Part 1}
		\item \href{https://www.youtube.com/watch?v=g3bH4JYmIpg&t=4s&ab_channel=TheOrganicChemistryTutor}{General Chemistry 1 Part 2}
		\item \href{https://www.youtube.com/watch?v=lSmJN1_uVpI&t=2s&ab_channel=TheOrganicChemistryTutor}{General Chemistry 2 Part 1}
		\item \href{https://www.youtube.com/watch?v=gOJaWfj1eoI&t=2s&ab_channel=TheOrganicChemistryTutor}{General Chemistry 2 Part 2}
		\item \href{https://ptable.com/?lang=en#Properties}{Periodic Table}
		\item \href{https://cpanhd.sitehost.iu.edu/C101webnotes/chemical-nomenclature/images/polyions.jpg}{Polyatomic Ions}
	\end{enumerate}
	
	\newpage
	
	\textbf{Notes: }
	\begin{enumerate}[label=\arabic*.]
		\item How many protons, electrons and neutrons are found in the ion shown below:
		\[ \,_{13}^{27}Al^{+3}\]
		\textbf{\boldmath This has 13 Protons, $13 - 3 = 10$ elections, and $27 - 13 = 14$ neutrons.}
		\item What is the correct name for the compound $N_2O_5$
		\begin{enumerate}[label=(\alph*)]
			\item Molecular Compounds - Nonmetal $\rightarrow$ Nonmetal
				\begin{enumerate}[label=(\roman*)]
					\item Prefixes - Mono-, Di-, Tri-, Tetra- for both nonmetals 
					\item $N_2O_5$ - Dinitrogen Pentoxide
					\item $SF_6$ - Sulfur Hexaflouride - (If first element is 1, then no need for -mono)
				\end{enumerate}
			\item Ionic Compounds - Metal $\rightarrow$ Nonmetal
				\begin{enumerate}[label=(\roman*)]
					\item Suffixes - -ide for Nonmetal
					\item $AlCl_3$ - Aluminum Chloride
					\item $MgF_2$ - Magnesium Fluoride
				\end{enumerate}
		\end{enumerate}
		\textbf{Dinitrogen Pentoxide}
		\item Calculate the percent composition of Aluminum in Aluminum Sulfite $Al_2(SO_3)_3$
		\\ \\
		We can find this by taking the mass of the Aluminum in the compound and dividing it with the entire mass of the compound:
		\[\frac{2(26.982)}{2(26.982) + 3(32.06) + 9(15.999)} = .183467 = 18.3467\%\]
		\item Nitrogen gas reacts with Hydrogen gas to form Ammonia.  Calculate the mass of the Ammonia($NH_3$) produced if 15g of Nitrogen gas reacts with excess Hydrogen gas.
		\[N_2 + 3H_2 \rightarrow 2NH_3\]
		Notice the following:
		\[15g N_2 \times \frac{1 \mol N_2}{2(14.007) g N_2} * \frac{2 \mol NH_3}{1 \mol N_2} * \frac{14.007 + 3(1.008)g NH_3}{1 \mol NH_3} = 18.2383 g NH_3\]
		\item 15g of Sodium Hydroxide is dissolved in enough water to produced a 250ml solution.  Calculate the molarity of the solution.
		\\ \\
		Notice that molarity is the ratio between moles of a compound to its volume that it diluted in.
		\[15g NaOH \times \frac{1 \mol NaOH}{22.990 + 15.999 + 1.008 g NaOH} * \frac{1}{250 * (10^{-3}) L NaOH} = 1.500 M NaOH \]
		\newpage
		\item How many mL of water must be added to a 200ml of a 0.75M solution of NaOH to dilute the concentration to 0.25M
		\[\frac{.75\mol\,NaOH}{1 L\,\,NaOH} \times \frac{.2}{.2} = \frac{.75(.2) \mol NaOH}{.2 L\,\,NaOH} = \frac{.75(.2) mol NaOH}{200 ml\,\,NaOH}\]
		Now we can see the following equality, where $x$ is how many ml to add to dilute the concentration to 0.25M
		\[\frac{.75(.2) \mol NaOH}{(200 + x) ml\,\,NaOH} = 0.25 M\]
		We can see the solution, $x$ below:
		\[\frac{.75(.2)\mol \,\, NaOH}{0.25 M} - 200ml = .6L - 200ml = 600ml - 200ml = 400ml = x\]  
		\item What is the correct oxidation state of Chromium in Sodium Dichromate ($Na_2Cr_2O_7$)
		\\ \\
		Notice the oxidation equation:
		\[2Na + 2Cr + 7O = 0\]
		We write out its charge into the equation and get:
		\[2(+1) + 2x + 7(-2) = 0 \qqarrow x = +6\]
		Notice the following rules about oxidations:
		\begin{enumerate}[label=(\alph*)]
			\item $H_2, N_2, O_2$ - Oxidation of 0
			\item $O = -2$ as an oxide, $O = -1$ as a peroxide
			\item $H = +1$ when bonded to a Nonmetal
			\item $H = -1$ when bonded to a Metal
		\end{enumerate}
		\item 38.6ml of a 0.249M NaOH solution was required to completely tritate 44.7ml of Sulfuric Acid ($H_2SO_4$) solution.  Determine the unknown concentration of the Sulfuric Acid solution.
		\[2NaOH + H_2SO_4 \rightarrow 2H_2O + Na_2SO_4\]
		Notice the following:
		\[38.6 ml\,\,NaOH \times \frac{1L}{10^3 ml} \times \frac{0.249 \mol\,\,NaOH}{1 L\,\,NaOH} \times \frac{1 \mol H_2SO_4}{2 \mol NaOH} \times \frac{1}{44.7 ml\,\,H_2SO_4} \times \frac{10^{3} ml}{1 L} = .108 M\,\,H_2SO_4\]
		\item A 250ml sample of Argon gas has a pressure of 1.25atm at a temperature of 300K.  Calculate the new pressure if the temperature is increased to 500K and the volume is decreased to 100ml.
		\\ \\
		Notice we can use the Combined Gas Law:
		\[\frac{P_1V_1}{T_1} = \frac{P_2V_2}{T_2}\]
		Applying it, we get:
		\[\frac{(1.25atm)(250ml)}{300K} = \frac{P_2(100ml)}{500K} \qqarrow P_2 = \frac{(1.25atm)(250ml)}{300K}\times\frac{500K}{100ml} = 5.2083 atm\]
		\item Calculate the density of Oxygen gas ($O_2$) at STP.
		\\ \\
		At STP, we know the temperature is $0^\circ C$, the pressure is 1atm and the volume of 1 mole of any gas is 22.4L.
		\\ \\
		Thus, we get the density of Oxygen gas at STP is as follows:
		\[\frac{2(15.999)\,\,g\,O_2}{22.4 L\,\, O_2} = \frac{1.42848\,g\,\,O_2}{L\,\,O_2}\]
		\\ \\
		Density can also be calculated the following way with $R$ being the gas constant, $0.08206\,L\,\text{atm}\mol^{-1}K^{-1}$, and $M$ being the molar mass:
		\[d = \frac{PM}{RT} = \frac{(1atm)(2(15.999) g\,\mol^{-1})}{(0.08206\,L\,\text{atm}\mol^{-1}K^{-1})(273 K)} = \frac{1.42833\,g\,\,O_2}{L\,\,O_2}\]
		\item Calculate the partial pressure of Ammonia ($NH_3$) if 24g of Nitrogen gas reacts with excess Hydrogen gas at 298K inside a 2.50L container.
		\[N_2 + 3H_2 \rightarrow 2NH_3\]
		So we can use the ideal gas law to calculate the partial pressure, $P$:
		\[PV = nRT  \qarrow P = \frac{nRT}{V}\]
		So we first need to calculate the number of moles of $NH_3$:
		\[n = 24g N_2 \times \frac{1\mol\,N_2}{2(14.007) g\,N_2} \times \frac{2\mol\,NH_3}{1\mol\,N_2} = \frac{24}{14.007}\mol\,NH_3\]
		Thus we get $P$:
		\[P = \frac{24 \mol\,NH_3}{14.007} \times \frac{0.08206\,L\,\text{atm} \,NH_3}{\mol K \,NH_3} * \frac{298K \,NH_3}{2.50L\,NH_3} = 16.75999 atm\]
		\item 4.722g of an unknown gas is collected over water inside a 2.75L container at 298K.  The total pressure inside the container is 749 torr and the vapor pressure of water is 23.76 torr.  Determine the identity of the unknown gas.
		\\ \\
		We can find the pressure of the unknown gas can be found by the following:
		\[P_{\text{total}} = P_{\text{uknown gas}} + P_{H_2O} \qarrow P_{\text{unknown gas}} = 749 - 23.76 = 725.24 \text{ torr} = 0.95426 \text{ atm}\]
		We can use the ideal gas law to find the number of moles to find the molar mass:
		\[PV = nRT \qarrow n = \frac{(0.95426 \text{ atm})(2.75L)}{(0.08206\,L\,\text{atm}\mol^{-1}K^{-1})(298K)} = 0.1073\mol\]
		Using $n$, we can see that the molar mass of the unknown gas is:
		\[M_{\text{unknown gas}} = 4.722g \div \frac{(0.95426 \text{ atm}) (2.75L)}{(0.08206 \,L\,\text{atm}\mol^{-1}K^{-1})(298K)} = 44.002195 \frac{g}{\mol}\]
		Looking at the choices between $H_2, CO_2, N_2, Xe$, we can see that 
		\[\textbf{\boldmath $CO_2$ has a molar mass of $(12.011 + 2(15.999)) = 44.009 \frac{g}{\mol}$}\]
		\newpage
		\item Which of the following statements is not correct?
		\begin{enumerate}[label=(\alph*)]
			\item The average kinetic energy of a sample of gas is dependent on temperature. 
			\\
			\textbf{\boldmath True - $KE = \frac{3}{2}RT$}
			\item The pressure inside a container is dependent on the total number of moles of gas particles inside the container.
			\item Heavier gas particles exert a greater pressure on the walls inside of the container.
			\item The average velocity of gas particles is dependent on temperature.
			\\
			\textbf{\boldmath True - $V_{rms} = \sqrt{\frac{3RT}{M}}$}
			\\
			This is the Root mean square Velocity, where $R = 8.3145 J\mol^{-1}K^{-1}$ and $M$ is the molar mass in $kg /\mol$
		\end{enumerate}
	\end{enumerate}

\end{document}
